\documentclass[9pt]{extarticle}
\usepackage[letterpaper, landscape, margin=.5in]{geometry}
\usepackage{multicol}
\usepackage{hyperref}
\pagestyle{empty}
\begin{document}
\centering\section*{Pyxam Cheat Sheet v0.3.5}
\begin{multicols}{2}
\raggedright\subsection*{Running Pyxam}
Usage \texttt{\$ pyxam [Options] template} \\
{\bf Command list:} \\
\begin{tabular}{l l l l}

help         & -h     &                & Show a list of options \\
version      & -v     &                & Show Pyxam's version number \\
format       & -f     & [format]       & Set export format \\
title        & -t     & [title]        & Set the title of the exam \\
alphabetize  & -a     &                & Enable lettered versioning \\
solutions    & -s     &                & Enable soultions \\
out          & -o     & [out]          & Set the output directory \\
tmp          & -tmp   & [tmp]          & Set the temporary directory \\
figure       & -fig   & [figure]       & Set the figure directory \\
population   & -p     & [population]   & Set class list \\
method       & -m     & [method]       & Set selection method for CSVs \\
number       & -n     & [number]       & Set the number of exams to generate \\
recomps      & -r     & [recomps]      & The number of LaTeX recompilations \\
shell        & -shl   & [shell]        & Set shell used to weave the exam \\
noweave      & -w     &                & Disable pweave \\
list         & -ls    &                & List all available formats \\
plugins      & -plg   &                & List all currently loaded plugins \\
htmltemplate & -htt   & [htmltemplate] & Specify an HTML template file \\
docs         & -docs  &                & Build Pyxam's docs for use locally \\
gitdocs      & -gdocs &                & Build Pyxam's docs for use on Github \\
api          & -api   &                & Run Pyxam in api mode \\
debug        & -d     &                & Disable file cleanup \\
logging      & -l     & [logging]      & Set the logging level for pyxam \\
             &        &                & 10: DEBUG \\
             &        &                & 20: INFO \\
             &        &                & 30: WARNING \\
             &        &                & 50: CRITICAL \\
\end{tabular} 

For more details see README.md \\
\subsection*{Inline Python Variables and Functions}
\begin{description}
\item\texttt{pyxam.number} \\
    The exam version number starting from 0
    \\
\item\texttt{pyxam.version} \\
    The exam version, either a number starting from 1 or a letter starting from A
    \\
\item\texttt{pyxam.student\_first\_name} \\
    The student's first name if available or a placeholder in the solutions document
    \\
\item\texttt{pyxam.student\_last\_name} \\
    The student's last name if available or a placeholder in the solutions document
    \\
\item\texttt{pyxam.student\_name} \\
    The student's full name if available or a placeholder in the solutions document
    \\
\item\texttt{pyxam.wildcard(min=None, max=None, set=None, n=pyxam.number, decimals=0)} \\
    Create a wildcard that can be used to pick from a list or generate a set of random numbers
    \\
\item\texttt{pyxam.import\_question(path)} \\
    Pastes the content of the file
    \\
\item\texttt{pyxam.args(args)} \\
    Set command line arguments from the document, not all options are useable
    \\
\item\texttt{pyxam.shuffle(choices)} \\
    Takes a list of strings and prints them out in a random order
    \\
\item\texttt{pyxam.numerical(solution, tolerance=0, percent=False)} \\
    Create a numerical question with a set tolerance
    \\
\item\texttt{pyxam.calculated(equation, tolerance=0, percent=False)} \\
    Creat a calculated question with a set tolerance, the equation must be provided as a string in moodle syntax
    \\
\item\texttt{pyxam.dataset(*wildcards)} \\
    Add a list of wildcards to the questions dataset so they can be used in moodle
    \\
\item\texttt{pyxam.categorize(course, category)} \\
    Questions will be added to the given course category in moodle
    \\
\end{description}
\subsection*{Examples}
\begin{description}
    \item See {\it examples/pyxam\_tex\_standard.tex} for an example tex file
    \item See {\it examples/pyxam\_org\_standard.org} for an example org file
    \item See {\it README.md} for a general overview of the tools and basic usage 
\end{description}
\subsection*{Emacs Shortlist}
\begin{tabular}{l l}
M-p & Previous shell command \\
C-x-C-v RET & Refresh the currently selected buffer \\
C-x-1 & Close all windows except the currently selected one \\
C-x-2 & Split window vertically \\
C-x-3 & Split window horizontally \\
C-x-0 & Close the currently selected window \\
C-x-k RET & Kill the currently selected buffer \\
\end{tabular}

\end{multicols}
\end{document}
