\documentclass[12pt]{exam}
    \usepackage[pdftex]{graphicx}
    \usepackage[T1]{fontenc}
    \catcode`\_=12
    \begin{document}
        \begin{questions}
            \titledquestion{Essay Question}
                This is the simplest question possible. It is just a prompt and has no solution.
            \titledquestion{Short Answer}
                Short answer questions are composed of a prompt and solution. When importing the question to Moodle the
                solution will be matched character for character.
                \begin{solution}
                    Place your solution in a solution block.
                \end{solution}
            \titledquestion{Multiple Choice}
                Multiple choice questions are made up of choices and correct choices. You can have as many of each as
                you need. All choices should appear in a choices block. By default multiple choice questions will be
                scrambled when imported to Moodle.
                \begin{choices}
                    \choice A
                    \choice B
                    \choice C
                    \CorrectChoice D
                \end{choices}
            \titledquestion{Multi Select}
                Multi select questions are a slight variation on multiple choice questions. By including multiple
                correct choices the Moodle format will automatically allow the user to select multiple answers.
                \begin{choices}
                    \choice A
                    \CorrectChoice B
                    \choice C
                    \CorrectChoice D
                \end{choices}
            \titledquestion{True or False}
                True or False questions are a variation of multiple choice questions where the only choices are True or
                False.
                \begin{choices}
                    \choice True
                    \CorrectChoice False
                \end{choices}
            \titledquestion{Numerical Question}
                Numerical questions look like short answer questions except they use the format $var=solution$ in the
                solution block. These questions will be automatically be converted to numerical questions when
                importing to moodle making the solution part of the format the answer.
                \begin{solution}
                    $y=4$
                \end{solution}
            \titledquestion{Numerical Question With Absolute Tolerance}
                Numerical questions can be given a tolerance which allows the user to enter a value within $\pm$ the
                tolerance of the solution.
                \begin{solution}
                    $y=4 tolerance 0.1$
                \end{solution}
            \titledquestion{Numerical Question With Percent Tolerance}
                Tolerance for numerical questions can also pe specified in percent.
                \begin{solution}
                    $y=4 tolerance 1\%$
                \end{solution}
            \titledquestion{Using a Plot in a Question}
                Plots can be added to any question be defining a comment block and then using the fig command.
<<echo=False, fig=True>>=
import matplotlib.pyplot
# To add a dataset we use the pyplot module in matplotlib and provide a dataset to the plot function
matplotlib.pyplot.plot([1,2,3,4])
# The axis of the plot can be labeled using the ylabel and xlabel functions
matplotlib.pyplot.ylabel('Y axis')
matplotlib.pyplot.xlabel('X axis')
# A title can be added using the title function
matplotlib.pyplot.title('A Simple Graph')
# Use the show function to finalize the figure and display it in the question
matplotlib.pyplot.show()
@
            \titledquestion{Using Random Parameters}
                You can define parameters in a python equation using random values in order to create a randomized
                question. We will use the block command to set up some variables.
<%
# Set our parameters
a = random.randint(0, 10)
b = random.randint(0, 10)
# Calculate our solution
c = a + b
%>
                Now that we've set up the question we can ask what is <%= a %> + <%= b %>?
                \begin{solution}
                    $<%= a %> + <%= b %> = <%= c %>$
                \end{solution}
            \titledquestion{Picking Parameters From a List}
                You can use python to pick random parameters from a defined list.
<%
# Define the lists for our parameters
list_a = [1, 2, 3, 4]
list_b = [1, 2, 3, 4]
# Use the choice function to select random parameters
a = random.choice(list_a)
b = random.choice(list_b)
# Calculate our solution
c = a + b
%>
                What is <%= a %> + <%= b %>?
                \begin{solution}
                    $<%= a %> + <%= b %> = <%= c %>$
                \end{solution}
            \titledquestion{Picking a Random Items From a List}
                For questions with solutions that are too difficulat to calculate in Python you can pick a dictionary
                from a list.
<%
# Define our list as list of dictionaries containing our parameters and solution
params = [{'a':1, 'b':2, 'c':3}, {'a':5, 'b':9, 'c':14}, {'a':0, 'b':7, 'c':7}]
# Select our parameters and solutions
param = random.choice(params)
# Set our parameters and solution
%>
                What is <%= a %> + <%= b %>?
                \begin{solution}
                    $<%= a %> + <%= b %> = <%= c %>$
                \end{solution}
            \titledquestion{Using the Exam Version to Pick Items From a List}
                If you want to pick a specific item in a list for the exam version you can use the pyxam object to
                access the exam number.
<%
# Define the lists for our parameters
list_a = [1, 2, 3, 4]
list_b = [4, 3, 2, 1]
# Use the exam number to select our parameters
a = list_a[pyxam.examnumber % len(list_a)]
b = list_b[pyxam.examnumber % len(list_b)]
# Calculate our solution
c = a + b
%>
                What is <%= a %> + <%= b %>?
                \begin{solution}
                    $<%= a %> + <%= b %> = <%= c %>$
                \end{solution}
            \titledquestion{Moodle Pick From a List}
                In order to generate a pick from a list question in Moodle you have to use a sepcific format. Variables
                have to be surrounded by curly brackets. You solution has to be given as an equation so here {a} + {b}
                can be written out verbatim. To provide potential parameters you need to provide items with the format
                where ${variable} [list of solutions]$ These questions still support tolerance.
                \begin{solution}
                    ${a} + {b} = {a} + {b} tolerance 0.01$
                    where ${a} [1.0, 7.5, 6.0]$
                    where ${b} [2.5, 4.0, 9.8]$
                \end{solution}
            \titledquestion{Constants}
                Constants can be referened in comments using the following commands.
                <%= pyxam.examnumber %> will be replaced with the exam number.
                <%= pyxam.examversion %> will be replaced with the exam version.
                <%= pyxam.studentname %> will be replaced with the student name if available.
                <%= pyxam.studentnumber %> will be replaced with the student number if available.
        \end{questions}
    \end{document}