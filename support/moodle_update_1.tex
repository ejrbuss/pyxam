% Compile this document to a pdf and produce solutions using the command:
%   $ pyxam -s -f pdf pyxam_tex_standard.tex
% Compile this document to moodle using the command:
%   $ pyxam -f moodle pyxam_tex_standard.tex
% Compile this document to html and produce solutions using the command:
%   $ pyxam -s -f html pyxam_tex_standard.tex
% Set the title of the output document to pyxam_standard
<%
pyxam.args('-t moodle_update_1')
%>
\documentclass[12pt]{exam}

    % The graphicx package is needed for figures and images
    \usepackage[pdftex]{graphicx}

    \begin{document}
        \begin{questions}

            \titledquestion{1. Wildcard Functions}
<<echo=False>>=
import math
A = pyxam.wildcard(set=[1, 2, 3, 4])
B = pyxam.wildcard(set=[25, 36, 49, 64])
# Wildcards can now be defined via function, here we define the function
def solution(a, b):
    return a + math.sqrt(b)
# Here we create the wildcard, note arguments a and b are members of the sets A and B
C = pyxam.wildcard(function=solution, args=[A, B])
@
                Find $ <%= {A} %> + \sqrt{<%= {B} %>} $. (brackets \textit{italics} and \textbf{bold} are now supported as well).
                \begin{solution}
                 <% pyxam.calculated('{C}', tolerance=20.0) %>
                \end{solution}
                <% pyxam.dataset(A, B, C) %>

        \end{questions}
    \end{document}
